\documentclass[twocolumn]{article}

\usepackage[T2A]{fontenc}          
\usepackage[english]{babel}
\usepackage[utf8]{inputenc}

\usepackage{amsmath}
\usepackage{subcaption}
\usepackage{mathtools}
\usepackage{graphicx}
\usepackage{color}
\usepackage{authblk}
\usepackage[colorlinks,citecolor=red,urlcolor=blue,bookmarks=false,hypertexnames=true]{hyperref}
\usepackage{geometry}
\geometry{
	a4paper,
	total={170mm,257mm},
	left=30mm,
	top=20mm,
}

\title{Информационный поиск\\ Обработка коллекции BY.WEB}
\author{Алфёров Василий, Швецова Анна, Казначеев Дмитрий}

\begin{document}

\maketitle        

\begin{abstract}
	Проблема ранжирования страниц в Интернете -- задача, поиском наиболее эффективного решения которой, вот уже на протяжении более двадцати лет занимаются ведущие IT-компании мира. В условиях динамически меняющихся огромных массивов данных важно использовать методы, позволяющие быстро реагировать на независимые изменения, чтобы подерживать ответы на запросы пользователей в актуальном состоянии. 
	
	Важным этапом в этом процессе является обработка данных документа, позволяющая сократить объем данных, необходимых для обучения ранжированию, и при этом сохранить информацию, содержающуюся в них в конденсированном состоянии. 
\end{abstract}

\section{Введение}

By.Web -- коллекция страниц в домене первого уровня BY (Беларусь). Коллекция была подготовлена компанией Яндекс в рамках мероприятий по оценке методов информационного поиска РОМИП. Использование коллекции регулируется \href{http://romip.ru/romip_agreement_for_nonparticipants2015_signed.pdf}{соглашением}. Коллекция предназначена исключительно для выполнения учебных заданий, распространять коллекцию или использовать ее в коммерческих целях запрещается.	

В рамках этого отчёта будут описаны следующие мероприятия по обработке и анализу данных коллекции By.Web:

\begin{enumerate}
	\item Первичная обработка коллекции
	\begin{enumerate}
    \item Распаковка коллекции из XML/base64, приведение к utf8
    \item Сбор метаинформации о каждом документе
    \item Удаление комментариев, CSS и JS
    \item Выделение контента
	\end{enumerate}
	\item Морфологическая обработка данных
	\begin{enumerate}
		\item Нормализация и токенизация текста 
		\item Постчёт метрик на корпусе
		\begin{enumerate}
			\item Статистика использования языков
			\item Средняя длина слова
			\item Доля стоп-слов в тексте
			\item tf-idf
			\item Зависимость частоты употребления от ранга слова в коллекции 
		\end{enumerate}
	\end{enumerate}
	\item Визуализация графа гиперссылок
	\begin{enumerate}
		\item Сбор гиперссылок на веб-страницах
		\item Очистка гиперcсылок
		\item Построение графа гиперссылок
		\item Визуализация графа
	\end{enumerate}
\end{enumerate}

\section{Первичная обработка данных}

Весь описываемый процесс происходит в \href{https://github.com/vasalf/hse-web-search-homework/blob/master/1/extract.py}{этом скрипте}.
Скрипт создаёт папку \texttt{extracted} и складывает туда по два файла для каждой страницы: текст и метаинформацию, последняя в формате JSON.
Кроме того, он создаёт отдельный файл для своей первичной статистики и пишет её туда в формате JSON.

\paragraph{Извлечение документов}

Документы лежат в XML-файлах. Для начала, нужно достать их оттуда.

Традиционных подходов к парсингу XML-документов два: DOM и SAX.
Первый проще и мощнее, второй заметно быстрее.
В стандартной библиотеке есть также и третий подход, называемый ElementTree.
Нам не хотелось разбираться с новыми подходами для настолько просто устроенных документов, поэтому мы выбирали между первыми двумя.

Проще всего выбрать DOM, но при этом есть опасность, что целое объектное дерево не поместится в оперативную память.
В таком случае следовало бы выбрать SAX или разобраться с ElementTree/pulldom.

В скрипте эта часть имеет заведённый под неё абстрактный класс — мы специально оставили место под написание SAX-парсера в случае, если самая простая реализация MiniDOM будет слишком ресурсоёмкой.

Эксперимент, однако, показал, что для каждого из десяти документов построенное MiniDOM'ом дерево занимало менее 1.5GiB в оперативной памяти, что при имеющихся у нас вычислительных ресурсах нас более чем устраивало. Соответственно, был оставлен MiniDOM-парсер, как самый простой вариант, не требующий чрезмерных человеческих и вычислительных ресурсов.

После извлечения контента нужно было преобразовать его в строчку. Мы использовали встроенный base64-декодер для получения массива байтов, а дальше перекодировали содержимое из cp1251 в UTF8 через встроенную библиотечку \texttt{codecs}. Причина последнего преобразования в том, что в языке Python гораздо проще работать именно с UTF8-строчками. Да и нам тоже на них приятнее смотреть.

\paragraph{Удаление HTML-разметки}

Мы использовали рекомендованную в тексте задания библиотечку BeautifulSoup.
В документации указан способ прикрутить туда альтернативные парсеры XML. Поскольку реальные данные из интернета вовсе не всегда являются корректными xml-документами, дефолтному парсеру от этих данных стало плохо.
После серии экспериментов мы останосились на \texttt{lxml}, дававшем более чем удовлетворительные результаты.

Из разметки мы извлекли метаинформацию и контент. Метаинформация содержит заголовок страницы, её URL (интересный факт, что не все урлы из коллекции являлись корректными ASCII-строчками), а также список урлов, на которые страница ссылается.

Контент можно доставать из документа, пройдясь по всем текстовым узлам, которые достал BeautifulSoup.
Однако не все узлы подойдут: на страницах есть ещё скрипты и CSS-стили, а также комментарии.
Если скрипты и стили можно отсеять по внешнему тегу, то с комментариями в каком-то смысле вышла беда.
По какой-то причине BeautifulSoup их считает за часть текста, не создавая под них отдельного тега.
Решения, основанные на регулярных выражениях, приводили либо к слишком большому времени обработки страницы, либо не были эффективными и оставляли различные артефакты.
В итоге мы смогли от большинства подобных артефактов избавиться через манипуляцию с параметрами BeautifulSoup, который всё же умеет делать элементы под теги, если правильно его попросить.

\paragraph{Параллелизм}

Обработка гигабайтов данных — непростая задача для питона.
Кроме того, из-за GIL там фактически недоступно многопоточное программирование без выхода в другие языки.
К счастью, первичная обработка идеально параллелится по данным, а значит, использовать многопроцессную архитектуру, по процессу на XML-файл, абсолютно ненакладно.
Мы запускали первичную обработку в 4 процесса и на нашем железе в таком режиме она занимала примерно 18 минут.

\paragraph{Первичная статистика}

Статистика, собранная после первичной обработки, приведена в таблице \ref{primary-stats}.

\begin{table}[h]
\begin{tabular}{|c|c|}
    \hline
        Количество документов                              & 200000 \\
        Средний размер документов в словах                 & 808 \\
        Средний размер документов в байтах                 & 7178 \\
    \hline
\end{tabular}
\caption{Статистика, собранная после первичной обработки}
\label{primary-stats}
\end{table}

Распределение длин документов в словах и в байтах, можно посмотреть на рисунках \ref{primary-length-w} и \ref{primary-length-b}, соотвтетсвенно, а соотношение объёма теста и исходного документа — на рисунке \ref{primary-volfraction}.

\begin{figure}[h]
    \includegraphics[width=.5\textwidth]{doc_words.png}
\caption{Распределение длин документов в словах}
\label{primary-length-w}
\end{figure}

\begin{figure}[h]
\includegraphics[width=.5\textwidth]{doc_lengths.png}
\caption{Распределение длин документов в байтах}
\label{primary-length-b}
\end{figure}

\begin{figure}[h]
\includegraphics[width=.5\textwidth]{doc_ratio.png}
\caption{Распределение отношения объёмов текста и исходного документа}
\label{primary-volfraction}
\end{figure}


\section{Оценка качества поиска}

Работа модели оценивалась на предоставленном наборе поисковых запросов на русском языке для коллекции BYWEB.2007.
\subsection{Метрики}

В качестве оценок использовались следующие метрики:

%  Ниже можно написать формулы для метрик.

\begin{itemize}
 	\item p@20
 	\item r@20
 	\item MAP@20
 	\item R-Precision
\end{itemize}

\subsection{Вариации}

Для сравнения эффективности разных методов обработки текста ответы на запросы были оценены на четырёх индексах: 

\begin{itemize}
	\item Исходные тексты документов
	\item Лемматизированные тексты документов
	\item Стемматизаированные тексты документов
	\item Исходные тексты с заголовсками
\end{itemize}

Для преобразования текстов использовался пайплайн, описанный в предыдущем отчёте. К нему была добавлена обёртка, записывающая файлы на диск после обработки и модуль, стемматизирующий текст.

Этот же подход был применен для обработки запросов. Для разбора и дампа формата xml была использована библиотека minidom.

Лемматизация производилась с использованием библиотеки \texttt{MyStem}, стемматизация -- \texttt{ntlk.stem.snowball.RussianStemmer}. Поиск по тексту с заголовками производился за счёт добавления оператора 'текст или заголовок' к поисковому запросу.

\subsection{Результаты оценок}

\begin{center}
	\begin{tabular}{l|cccc}
		& p@20 & r@20 & MAP@20 & R-Precision\\
		\hline
		Исходные & 0.31 & 0.19 & 0.45 & 0.25\\
		Лемматизация & {\bf 0.36} & {\bf 0.24} & {\bf 0.49} & {\bf 0.3}\\
		Стемминг & 0.34 & 0.22 & 0.47 & 0.28 \\
		Заголовки & 0.32 & 0.19 & 0.46 & 0.26\\
	\end{tabular}
\end{center}

Как мы видим, лемматизация работает лучше, чем стемминг, а стемминг лучше, чем тексты без обработки. 

Рассмотрим примеры, на которых различие в качестве поиска по нашим метрикам максимально (по метрике F1):

\begin{itemize}
	\item Лемматизация vs Исходный текст (на всех запросах лемматизация отработала лучше, $\Delta F1 \approx 0.25-0.27$)\begin{itemize}
		\item ''ОБОИ НА РАБОЧИЙ СТОЛ'' vs ''обои на рабочий стол''. Довольно тривиальный запрос, который в исходном варианте за счёт наличия капса терял много хороших кандидатов. После обработки текста проблема пропала естественным образом.
		\item ''почему мышцы спины в тонусе и как с этим справиться'' vs ''почему мышца спина в тонус и как с это справляться'' -- много слов в запросе в неочевидных падежах, которые плохо находились в ненормализованном тексте. 
		\item ''строительство конюшни'' vs ''строительство конюшня'' -- редкое слово, ещё и в падеже.
	\end{itemize}
	\item Стемминг vs Исходный текст (на всех запросах стемминг отработал лучше, $\Delta F1 \approx 0.26-0.27$) \begin{itemize}
		\item ''еврейские детские песни'' vs ''еврейск детск песн'' -- стемминг помог избавиться от числа.
		\item ''реле включения кондиционера Пежо 406'' vs ''рел включен кондиционер пеж'' в процессе ушло число, которое, на самом деле, имело значение в запросе, но поскольку у нас не очень большое количество документов, то мы выграли, получив больше документов про пежо.
		\item ''ОБОИ НА РАБОЧИЙ СТОЛ'' vs ''обои на рабочий стол''. Та же проблема, что и при сравнении с лемматизацией.
	\end{itemize}
\end{itemize}

\section{Построение и визуализация графа гиперссылок}
\paragraph{}
Процесс построения графа является одним из этапов конвеера обработки и происходит в \href{https://github.com/vasalf/hse-web-search-homework/blob/master/1/pipeline/graph_builder.py}{этом скрипте}. Скрипт создаёт в папке \textbf{result} файлы \textbf{graph\_full.gml} и \textbf{graph\_top.gml}. В файлах в формате GML (Graph Modelling Language) содержатся описания целого графа гиперссылок и подграфа из 150 наиболее популярных страниц.
\paragraph{Построение графа}

Этап построения графа является этапом конвеера обработки данных.

 При инициализации, создаётся пустой ориентированный граф, ассоциативные массивы из URL в целочисленные индексы и обратно, а так же пустое множество обработанных URL обработанных страниц. Для работы с графами использована библиотека \textbf{networkx}, выбранная из-за удобства создания и изменения графа и наличия графовых алгоритмов, в т.ч. PageRank.
 
 На этапе обработки веб-страницы на конвеере мы принимаем выделенную на предыдущем этапе метаинформацию о странице $i$, в которой содержится URL страницы и список URL всех гиперссылок со страницы. URL страницы нормализуется с помощью библиотеки \textbf{urllib}, добавляется в множество обработанных страниц, и ему присваивается уникальный индекс $p_i$. 
 
 Затем каждая гиперссылка $j$ со страницы нормализуется с помощью \textbf{urljoin} относительно базового URL страницы, и из полученного URL убираются все URL-фрагменты. К примеру, на странице \texttt{http://tut.by} гиперссылка \texttt{/index.php\#top} преобразуется в  \texttt{http://tut.by/index.php}. Затем ей присваивается уникальный индекс $p_j$ и в граф добавляется ребро $p_i \rightarrow p_j$.
 
 На этапе сохранения результата весь граф сериализуется в формат GML и сохраняется в \textbf{result/graph\_full.gml}. После этого на графе запускается алгоритм PageRank, и  в \textbf{result/graph\_top.gml} сохраняется подграф на 150 вершинах с максимальным рейтингом PageRank.

\end{document}
