\section{Построение и визуализация графа гиперссылок}
\paragraph{}
Процесс построения графа является одним из этапов конвеера обработки и происходит в \href{https://github.com/vasalf/hse-web-search-homework/blob/master/1/pipeline/graph_builder.py}{этом скрипте}. Скрипт создаёт в папке \textbf{result} файлы \textbf{graph\_full.gml} и \textbf{graph\_top.gml}. В файлах в формате GML (Graph Modelling Language) содержатся описания целого графа гиперссылок и подграфа из 150 наиболее популярных страниц.
\paragraph{Построение графа}

Этап построения графа является этапом конвеера обработки данных.

 При инициализации, создаётся пустой ориентированный граф, ассоциативные массивы из URL в целочисленные индексы и обратно, а так же пустое множество обработанных URL обработанных страниц. Для работы с графами использована библиотека \textbf{networkx}, выбранная из-за удобства создания и изменения графа и наличия графовых алгоритмов, в т.ч. PageRank.
 
 На этапе обработки веб-страницы на конвеере мы принимаем выделенную на предыдущем этапе метаинформацию о странице $i$, в которой содержится URL страницы и список URL всех гиперссылок со страницы. URL страницы нормализуется с помощью библиотеки \textbf{urllib}, добавляется в множество обработанных страниц, и ему присваивается уникальный индекс $p_i$. 
 
 Затем каждая гиперссылка $j$ со страницы нормализуется с помощью \textbf{urljoin} относительно базового URL страницы, и из полученного URL убираются все URL-фрагменты. К примеру, на странице \texttt{http://tut.by} гиперссылка \texttt{/index.php\#top} преобразуется в  \texttt{http://tut.by/index.php}. Затем ей присваивается уникальный индекс $p_j$ и в граф добавляется ребро $p_i \rightarrow p_j$.
 
 На этапе сохранения результата весь граф сериализуется в формат GML и сохраняется в \textbf{result/graph\_full.gml}. После этого на графе запускается алгоритм PageRank, и  в \textbf{result/graph\_top.gml} сохраняется подграф на 150 вершинах с максимальным рейтингом PageRank.