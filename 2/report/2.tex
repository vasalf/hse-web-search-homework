\section{Оценка качества поиска}

Работа модели оценивалась на предоставленном наборе поисковых запросов на русском языке для коллекции BYWEB.2007.
\subsection{Метрики}

В качестве оценок использовались следующие метрики:

%  Ниже можно написать формулы для метрик.

\begin{itemize}
 	\item p@20
 	\item r@20
 	\item MAP@20
 	\item R-Precision
\end{itemize}

\subsection{Вариации}

Для сравнения эффективности разных методов обработки текста ответы на запросы были оценены на четырёх индексах: 

\begin{itemize}
	\item Исходные тексты документов
	\item Лемматизированные тексты документов
	\item Стемматизаированные тексты документов
	\item Исходные тексты с заголовсками
\end{itemize}

Для преобразования текстов использовался пайплайн, описанный в предыдущем отчёте. К нему была добавлена обёртка, записывающая файлы на диск после обработки и модуль, стемматизирующий текст.

Этот же подход был применен для обработки запросов. Для разбора и дампа формата xml была использована библиотека minidom.

Лемматизация производилась с использованием библиотеки \texttt{MyStem}, стемматизация -- \texttt{ntlk.stem.snowball.RussianStemmer}. Поиск по тексту с заголовками производился за счёт добавления оператора 'текст или заголовок' к поисковому запросу.

\subsection{Результаты оценок}

\begin{center}
	\begin{tabular}{l|cccc}
		& p@20 & r@20 & MAP@20 & R-Precision\\
		\hline
		Исходные & 0.31 & 0.19 & 0.45 & 0.25\\
		Лемматизация & {\bf 0.36} & {\bf 0.24} & {\bf 0.49} & {\bf 0.3}\\
		Стемминг & 0.34 & 0.22 & 0.47 & 0.28 \\
		Заголовки & 0.32 & 0.19 & 0.46 & 0.26\\
	\end{tabular}
\end{center}

Как мы видим, лемматизация работает лучше, чем стемминг, а стемминг лучше, чем тексты без обработки. 

Рассмотрим примеры, на которых различие в качестве поиска по нашим метрикам максимально (по метрике F1):

\begin{itemize}
	\item Лемматизация vs Исходный текст (на всех запросах лемматизация отработала лучше, $\Delta F1 \approx 0.25-0.27$)\begin{itemize}
		\item ''ОБОИ НА РАБОЧИЙ СТОЛ'' vs ''обои на рабочий стол''. Довольно тривиальный запрос, который в исходном варианте за счёт наличия капса терял много хороших кандидатов. После обработки текста проблема пропала естественным образом.
		\item ''почему мышцы спины в тонусе и как с этим справиться'' vs ''почему мышца спина в тонус и как с это справляться'' -- много слов в запросе в неочевидных падежах, которые плохо находились в ненормализованном тексте. 
		\item ''строительство конюшни'' vs ''строительство конюшня'' -- редкое слово, ещё и в падеже.
	\end{itemize}
	\item Стемминг vs Исходный текст (на всех запросах стемминг отработал лучше, $\Delta F1 \approx 0.26-0.27$) \begin{itemize}
		\item ''еврейские детские песни'' vs ''еврейск детск песн'' -- стемминг помог избавиться от числа.
		\item ''реле включения кондиционера Пежо 406'' vs ''рел включен кондиционер пеж'' в процессе ушло число, которое, на самом деле, имело значение в запросе, но поскольку у нас не очень большое количество документов, то мы выграли, получив больше документов про пежо.
		\item ''ОБОИ НА РАБОЧИЙ СТОЛ'' vs ''обои на рабочий стол''. Та же проблема, что и при сравнении с лемматизацией.
	\end{itemize}
\end{itemize}