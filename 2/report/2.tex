\section{Оценка качества поиска}

Работа модели оценивалась на предоставленном наборе поисковых запросов на русском языке для коллекции BYWEB.2007.
\subsection{Метрики}

В качестве оценок использовались следующие метрики:

%  Ниже можно написать формулы для метрик.

\begin{itemize}
 	\item p@20
 	\item r@20
 	\item MAP@20
 	\item R-Precision
\end{itemize}

\subsection{Вариации}

Для сравнения эффективности разных методов обработки текста ответы на запросы были оценены на четырёх индексах: 

\begin{itemize}
	\item Исходные тексты документов
	\item Лемматизированные тексты документов
	\item Стемматизаированные тексты документов
	\item Исходные тексты с заголовсками
\end{itemize}

Для преобразования текстов использовался пайплайн, описанный в предыдущем отчёте. К нему была добавлена обёртка, записывающая файлы на диск после обработки и модуль, стемматизирующий текст.

Этот же подход был применен для обработки запросов. Для разбора и дампа формата xml была использована библиотека minidom.

Лемматизация производилась с использованием библиотеки \texttt{MyStem}, стемматизация -- \texttt{ntlk.stem.snowball.RussianStemmer}. Поиск по тексту с заголовками производился за счёт добавления оператора 'текст или заголовок' к поисковому запросу.

\subsection{Результаты оценок}

\begin{center}
	\begin{tabular}{l|cccc}
		& p@20 & r@20 & MAP@20 & R-Precision\\
		\hline
		Исходные & 0.31 & 0.19 & 0.45 & 0.25\\
		Лемматизация & {\bf 0.36} & {\bf 0.24} & {\bf 0.49} & {\bf 0.3}\\
		Стемминг & 0.34 & 0.22 & 0.47 & 0.28 \\
		Заголовки & p@20 & r@20 & MAP@20 & R-Precision\\
	\end{tabular}
\end{center}

Как мы видим, лемматизация работает лучше, чем стемминг, а стемминг лучше, чем тексты без обработки. 
Почему обработка улучшает результат, довольно очевидно: уменьшается количество фловоформ. Разберёмся, в чём разница между стеммингом и лемматизацией.

Основная разница заключается в том, что лемматизация учитывает контекст и сохраняет часть речи слова. 
