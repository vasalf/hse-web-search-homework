\section{Индексация коллекции}

Для индексации коллекции в Elasticsearch мы использовали официальные \href{https://elasticsearch-py.readthedocs.io/en/master/api.html}{бинды} в питоне. Несмотря на некоторые интересные спецэфеекты и несколько устаревшую документацию, в целом эта библиотека справляется посылать HTTP-запросы не хуже \texttt{curl} и даже немного упрощает жизнь в определённые моменты времени.

Мы индексировали plain-text данные, полученные при выполнении первого ДЗ. На наш взгляд, у них было достаточно хорошее качество фильтрации тегов.

В индекс мы положили тело документа и его заголовок для использования в дальнейших заданиях. Маппинг в индесе выглядит следующим образом:

\begin{verbatim}
{
  "mapping": {
    "properties": {
      "body": {
        "type": "text"
      },
      "title": {
        "type": "text"
      },
      "url": {
        "type": "text",
        "fields": {
          "keyword": {
            "type": "keyword",
            "ignore_above": 256
          }
        }
      },
      "pagerank": {
	      "type": "rank_feature"
      }
    }
  }
}
\end{verbatim}

Для ускорения индексации все документы заливались в одном куске с помощью bulk API, предоставленным Elasticsearch. Согласно документации этой функции в питоновских биндах, на самом деле запросы разбивались на блоки по 500.

В локальную копию Elasticsearch все запросы заливались за 120 секунд. Индекс занимал ~1.2G (число было получено через специальную команду питоновских биндов).

В нашей команде была проблема с оборудованием — ноут лишь одного участника справился обработать все данные. Чтобы другие участники имели доступ ко всем данным, было решено разместить их в облаке Elasticsearch.

Загрузка в облако заняла ~4400 секунд и индекс в облаке занимает ~1.8G.
Первое объясняется тем, что сеть со стороны Google Cloud загружена больше, чем наши локальные лупбеки. Второе объясняется, по всей видимости, разными версиями Elasticsearch локально и в облаке.

Стоит заметить, что после обработки многих запросов от всех участников команды вес индекса в облаке возрос до ~3.6G. Видимо, это связано с кешами и логами.

Также нам необходимо было решить вопрос с безопасностью в облаке. Нельзя хранить пароль от облака в коде в репозитории. Проблема известная и в разных местах в реальной жизни решается по-разному. Чтобы не усложнять себе жизнь и код, каждый участник сохранял у себя локально файлик \texttt{credentials.txt}, в котором лежат хост, логин и пароль. От греха файл был добавлен в \texttt{.gitignore}.
